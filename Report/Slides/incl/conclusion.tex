\section{Wrap Up}

\begin{frame}
	\frametitle{Strengths and Caveats: Serial}
	\begin{itemize}
		%Pro
		\item[$\bullet$] Recursive calls are avoided
		\item[$\bullet$] No class inheritance
		\item[$\bullet$] Tree update method to avoid full reconstruction
		\pause
		%Neg
		\item[$\circ$] Non-local memory allocations, pointers
		\item[$\circ$] Tree update likely looses efficiency for large time steps
		\item[$\circ$] Poor management of ``extreme'' scenarios (loss of all particles)
	\end{itemize}
\end{frame}

\begin{frame}[fragile]
\frametitle{Strengths and Caveats: Parallel}
	\begin{itemize}
		%Pro
		\item[$\bullet$] Transmitting all particles avoids complex routines (tree completion)
		\item[$\bullet$] MPI type provided for sending particles (no pointers transmitted)
		\item[$\bullet$] Duplicated work (tree building, updates) instead of communication time
		%Neg
		\item[$\circ$] Large transfers for large system sizes ($\sim\SI{1}{\giga\byte}$ for $10^7$ particles!)
		\item[$\circ$] No load balancing (incl. removed particles), high spin time
		\item[$\circ$] \lstinline|IsLeaf| method ``fails'' in the parallel version
	\end{itemize}
\end{frame}

\begin{frame}
\frametitle{Improvements}
	The following steps could still be conducted
	\begin{itemize}
		\item Managing error-prone scenarios (e.g. loss of all particles)
		\item Verifying whether a tree update is efficient on very large/dense trees
		\item Tree completion methods to avoid large transfers
		\item Load balancing between threads
		\item MPI-IO
	\end{itemize}
\end{frame}