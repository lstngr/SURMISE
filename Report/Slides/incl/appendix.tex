% Top secret section with extra material for a possible FAQ session at the end.
\appendix
\setbeamertemplate{section in toc}{\inserttocsection}
\metroset{sectionpage=none}

\begin{frame}
	\frametitle{Appendix}
	\tableofcontents
\end{frame}

\begin{frame}[fragile=singleslide]
\section{Implementation of BH from the FMM}
\frametitle{Implementation of BH from the FMM}
From the Fast Multipole (v0.5) to Barnes-Hut (v0.7), starting \formatdate{26}{05}{2019} (pre-release).
\begin{lstlisting}
include/iomanager.hpp    |   28 ++++
include/nodes.hpp        |  122 ++++++---------
include/sroutines.hpp    |   30 ++++
include/stypes.hpp       |   22 ++-
src/config_file.cpp      |    4 +-
src/iomanager.cpp        |  134 ++++++++++++++++
src/main.cpp             |    9 +-
src/nodes.cpp            | 1118 ++++++++++++++++...
src/sinput.cpp           |    2 +-
src/sroutines.cpp        |  129 +++++++++++++++
src/stypes.cpp           |   63 +++++++-
22 files changed, 1942 insertions(+), 882 deletions(-)
\end{lstlisting}
\end{frame}


\begin{frame}
	\section{Recursive Methods in the FMM}
	\frametitle{Recursive Methods in the FMM}
	All methods in the Barnes-Hut code avoid recursive calls.
	\begin{itemize}
		\item Much deeper levels possible for unbalanced problems. Stack could become large.
		\item More conditional statements, but also removes class inheritance (and thus virtual calls, vtables).
	\end{itemize}
\end{frame}

\begin{frame}[t]
	\section{Recursive Methods in the FMM (Cont.)}
	\frametitle{Recursive Methods in the FMM (Cont.)}
	Up to v0.5 (FMM attempt), code was written with recursive instructions (and class inheritance). With \only<1>{\lstinline|O0|}\only<2>{\lstinline|O3|},
	\begin{columns}
		\begin{column}{0.7\textwidth}
			\only<1>{\lstinputlisting[basicstyle=\tiny]{incl/perf.o0.hist.0}}
			\only<2>{\lstinputlisting[basicstyle=\tiny]{incl/perf.o3.hist.0}}
		\end{column}
		\begin{column}{0.3\textwidth}
			$10^5$ particles.
			
			\only<1>{\SI{49.92}{\second} per iter.}\only<2>{\SI{6.45}{\second} per iter.}
			
			\only<1>{3 iterations sampled:\SI{149.75}{\second}}\only<2>{3 iterations sampled:\SI{129.07}{\second}}
		\end{column}
	\end{columns}
	\only<2>{The compiler ``mysteriously'' destroys the recursive calls with enough optimization.}
\end{frame}

\begin{frame}
	\section{Why ``SuRMISe''}
	\frametitle{Why ``SuRMISe''}
	\textbf{Surmise} \footcite{surmise}
	\begin{quote}
		to guess something, without having much or any proof
	\end{quote}
	\par
	\vspace{1cm}
	Also the acronym for ``\alert{Su}rely \alert{R}uining \alert{M}y \alert{I}mpossible \alert{Se}mester''.
\end{frame}

\begin{frame}[fragile]
	\section{Benchmarking Configuration}
	\frametitle{Benchmarking Configuration}
	Benchmarks for the \lstinline|clusters| and \lstinline|vlargeuniform| sets run with
\begin{lstlisting}
npart=1000000
tevol_dt=0.00100000000
theta=0.400000
max_iter=10,20
\end{lstlisting}
	Compiled with \lstinline|O3|, modules \lstinline|intel, intel-mpi|. Execution time measured with \lstinline|time|, profiling with Intel Amplifier.
\end{frame}

\begin{frame}
	\section{Problems affecting the code}
	\frametitle{Problems affecting the code}
	\begin{itemize}
		\item When computing a leaf's interaction with a center of mass in an upstream branch (in which the leaf is located), the leaf's contribution to the upstream center of mass is not removed (that is, self-interaction occurs).
		\item Cases where only on particle remains in the system are not handled.
	\end{itemize}
\end{frame}